\chapter{The Problem}
\setcounter{page}{1}
\label{chapter0}
\section{The Problem}
The initial problem that was stated was to develop a method for a Baxter robot to run a sweet shop to serve people. Ideally, there would be multiple steps in this interaction. Firstly a customer would walk up in front of Baxter and ask for a certain number and type of sweet. The robot would then go into a bowl of mixed sweets and retrieve the requested ones and give them to the customer. The robot would then ask for a certain amount of money for the sweets, take money from the customer and give them the correct change back.
To further clarify this problem, we need to look into the different aspects of this project and how they can fit together to build up a coherent solution. In robotics, there are multiple different challenge areas, such as machine vision, machine learning, object manipulation and human-robot interaction. These areas can all be applied to the problem at different levels, which be explored further throughout the project.
\section{Approach}
The Herb 2.0 paper (2) separates the robot's architecture into several main components that make up the overall system: Robot Behaviour, Manipulation Planning, Planning Under Clutter and Uncertainty, Interacting with People and Perception. Initially, these sections can be used as a guide for developing the overall system architecture for Baxter.
\newline\newline
Firstly, manipulation planning will be an important and complex task in the sweet shop. Baxter will have to be able to grasp a specified number and type of sweet and give it to a customer in some way or another. The manipulation of the sweets will be a key task in this project, allowing the robot to use some methods of separating the sweets from a bowl into the required amount. This could be done with multiple manipulation methods - picking the sweets up and dropping them on the table and separating them out (3), scooping some sweets up and looking at the sweets in Baxter's grasp at the time. This challenge will most likely have to test multiple methods and see which is the most efficient/most appropriate for the project. Further challenges could be met in manipulation if the project gets to a point where Baxter can manipulate money. Notes provide a manipulation challenge in the fact that they are soft, which means it will be hard to develop a method to grasp and separate notes. This manipulation could be taken further and then taken into looking at the current areas of research here, such as the approach to robotic towel folding using cloth grasping (4).
\newline\newline
Perception will be a key task in the project. Baxter will have to be able to look at his environment and eventually, be able to recognise a bowl, the individual sweets within a bowl, the table on which the objects are stored, the human purchasing sweets and some form of money/currency. Since Baxter uses a Kinect 2.0, object recognition will be attempted with point clouds initially, although there are some other methods that can be looked into (2D methods etc). There will be a challenge in learning how to analyse and manipulate 3D point clouds to perform vision tasks and getting Baxter to recognise the many different objects. This concept could further be expanded upon if there is time to include clutter in the environments and allowing Baxter to plan manipulation around the clutter using algorithms as this is a key area of robotics. The computer vision module from year 3 should be a good building block for this task, though working with 3D analysis was covered in very brief detail.
\newline\newline
The complexity of the Human Interaction probably will depend on the time left at the end of the project (as it would first be key for Baxter to be able to recognise and manipulate sweets). For human interaction, methods could be implanted such as Kinect's gesture recognition for pointing at things or basic voice recognition commands. A nice implementation of the sweet shop would at least include the ability for Baxter to give the sweets or change directly to the person (5). This area could be expanded into many challenging areas, such as allowing Baxter to have basic conversations with the customer, understand more complex commands etc.