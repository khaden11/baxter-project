\chapter{Background And Literature Review}
\label{chapter2}
The idea of this project is to produce a fully functioning piece of software that allows Baxter to converse with customers in a sweet shop environment. To do this, background research was taken into the main areas I would need knowledge of in this project: robotic systems, machine vision, manipulation and human interaction. To develop a robotic architecture, it was necessary to look at some other robotics systems to see the main steps they took to plan and build them. The aspects discussed in the approach above were further researched below.
\section{Robotic Behaviour}
Robots use a behaviour engine to model their reactions to the environment in a logical way. This simple logic can be transferred to the shopkeeper project by Baxter having different behaviours to carry out in the shop. Those could be listening to a customer's command or picking up a particular sweet. All these behaviours and the logic along with them should be developed for make a robust set of behaviours for Baxter to use.
\newline\newline
Two main architectures exist to develop robotic behaviour have been used in multiple robotic systems: a classical approach and a subsumption architecture [\textbf{REF}]. A classical architecture is based on a pipeline-like approach, where first perception occurs, building up a model of the world, then from that model, planning is made then tasks are executed. A subsumption architecture works in more of a layered model, where multiple tasks run at the same time, so the layers act in parallel. That means that tasks such as exploration and wandering would run together, both with access to the robot's motors.
\newline\newline
The simplest approach to the shopkeeper's architecture seems to be a classical pipeline approach. It doesn't make sense for a subsumption architecture, since Baxter will need to re-analyse his environment before planning any manipulation tasks. Therefore a basic logical behaviour architecture could be theorised for a simple task of picking up sweets. In an example task, Baxter would want to pick up a sweet from a pile of sweets on the table. This could be split into a logical order of behaviours: recognise the sweets on the table, plan which sweet he wants to pick up, then pick up the sweet. This has to be done in this order, as Baxter will need to build up a model of his environment before he can pick the sweet up.
\section{Machine Vision}
Machine vision is the area of computer science where images can be analysed and understood. This is often used in robotics to help robots recognise and understand objects in their environment so they can react and respond appropriately. In the sweet shop, Many robotic vision systems are used in conjunction with manipulation and other aspect's of the robot's behaviour, so a robot will recognise and understand it's surrounding and interpret that in real-time to feed the information back to other processes. For example, this technique can be applied to Baxter looking at the table and then telling the manipulation process where the sweet is located to pick it up.
\subsection{Image Processing Techniques}
There have been multiple approaches to object detection in machine vision. The approaches analyse pixels by segmenting images into the desired edges/distinguishing features. 
\newline\newline\textbf{CONTOUR PARAGRAPH}
\newline\newline\textbf{TEMPLATE MATCHING}
\subsection{Pointcloud Analysis}
Vision techniques used in a 3D Pointcloud have a slightly different approach. Instead of segmenting and processing the pixels in an image, object analysis is more concentrated around grouping of points that are close in orientation and colour.
\newline\newline\textbf{RANSAC SEARCH METHODS}
\newline\newline\textbf{3D MODEL MATCHING}
\section{Manipulation Planning}
Manipulation planning is a key aspect of robotic systems, where the hardware of the robot and the limitations need to be considered before integrating manipulation techniques. Manipulation tasks in the sweet shop include tasks such as picking up sweets, giving them to the customer, which will all require planning for the position and movement of the grippers.
\subsection{Inverse Kinematics}
Inverse kinematics is one of the common movement planning techniques in robotics. This involves a kinematic analysis of the current state of the robot's system, looking at the all of the joint positions and angles. Normal kinematics works out how and where to move the limbs with a specified range and then works out the endpoint of that limb. Inverse kinematics does the reverse of this where you specify a coordinate endpoint of the limb or specify joint angles, inverse kinematics will use some mathematics to calculate the limb movement from the initial position to the given one. Baxter's inverse kinematics requires an x, y, z coordinate for an endpoint, along with a quaternion for the rotation of the arm.
\newline\newline
Some of the main issues with using inverse kinematics are that mainly, there are multiple solutions to get the limb to the specified position and one of those solutions has to be chosen. The most obvious solution to choose would be to select the one which contains the least overall movement to get to the right location. The problem with the multiple solutions is that sometimes, unexpected routes can be taken instead of the desired one. This usually tends to occur because a perceived human solution to a joint position will not be the closest solution in the joint's coordinate system, rather than the world coordinate system.[\textbf{REF}]
\section{Planning Under Clutter}
Planning under clutter could be an important aspect of the project depending on how the sweets are placed on the table. If they are grouped together, Baxter will have to plan to separate them from the clutter. Also other objects on the shop's counter could cause clutter issues with recognition and manipulation planning.
\section{Interacting with People}
Human interaction is going to be key to making the robot shopkeeper a realistic and comprehensive experience. By testing with human customers, this system could be made robust and efficient. This will be important once for Baxter to understand the customer's request and interact with them accordingly in the shop.
\subsection{Voice Recognition}
\subsection{Skeleton Recognition}